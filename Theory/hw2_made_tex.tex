\documentclass[a4paper,12pt]{article}
\pagestyle{empty}
\usepackage[T2A]{fontenc}
\usepackage[utf8]{inputenc}
\usepackage[russian]{babel}
\usepackage{cmap}
\usepackage{amsthm}
\usepackage{amsmath}
\usepackage{units}
\usepackage{fancyhdr}
\usepackage{forloop}
\usepackage{amssymb}
\usepackage{url}
\usepackage{hyperref}
\usepackage{xcolor}
\usepackage{color}
\usepackage{pifont}
\usepackage{graphicx}
\usepackage{subfig}
\definecolor{urlcolor}{HTML}{0645AD}
\hypersetup{pdfstartview=FitH, linkcolor=black,urlcolor=urlcolor, colorlinks=true}
\usepackage{algorithm}
\usepackage{algpseudocode}

\usepackage{tikz}

\usepackage{verbatim}
\usetikzlibrary{arrows,shapes}

\theoremstyle{definition}

\newtheorem{theorem}{Теорема}
\newtheorem{lemma}{Лемма}
\newtheorem{definition}{Определение}
\newtheorem{corollary}{Следствие}
\newtheorem{example}{Пример}
\newtheorem{remark}{Замечание}
\newtheorem{exercise}{Упражнение}

\renewcommand{\thesection}{\arabic{section}}

\renewcommand{\headrulewidth}{0.4pt}
\renewcommand{\footrulewidth}{0.4pt}

\fancyhead[RO]{\bfseries \rightmark}
\fancyhead[LE]{\bfseries \rightmark}
\fancyhead[LO, RE]{Методы оптимизации в машинном обучении}
\fancyfoot[L]{MADE 2020}
%For multipage documents only!
\fancyhead[R]{Страница \thepage}
%Uncomment this for 1-page sheets
\fancyfoot[R]{}
\fancyfoot[C]{}

\pagestyle{fancy}

\renewcommand{\baselinestretch}{1.0}
\renewcommand\normalsize{\sloppypar}

\setlength{\topmargin}{-0.1in}
\setlength{\textheight}{9.1in}
\setlength{\oddsidemargin}{-0.3in}
\setlength{\evensidemargin}{-0.3in}
\setlength{\textwidth}{7in}
\setlength{\parindent}{0ex}
\setlength{\parskip}{1ex}

\newenvironment{comm}{}{}

\newcounter{problemset}
\newcounter{totalpages}
%Here you should set the total number of pages
\setcounter{totalpages}{1}

\def \topic {Домашнее задание №2. \\Стохастическая оптимизация.\\ Методы маломерной оптимизации. Градиентный спуск}

\def \ZZ {\mathbb Z}
\def \FF {\mathbb F}
%\def \RR {\mathbb R}
\def \QQ {\mathbb Q}
\def \NN {\mathbb N}
\def \PP {\mathbb P}
\def \EE {\mathbb E}
\def \DD {\mathbb D}
\def \CC {\mathbb C}

\def\cA{{\cal A}}
\def\cB{{\cal B}}
\def\cD{{\cal D}}
\def\cC{{\cal C}}
\def\cQ{{\cal Q}}
%\def\R{{\cal R}}
\def\cM{{\cal M}}
\def\cN{{\cal N}}
\def\cT{{\cal T}}
\def\cP{{\cal P}}
\def\cF{{\cal F}}

\def\cE{{\mathcal{E}}}

\def\poly{{\text{poly}}}

\newcommand{\argmin}{\mathop{\arg\!\min}}
\newcommand{\circledOne}{\text{\ding{172}}}
\newcommand{\circledTwo}{\text{\ding{173}}}
\newcommand{\circledThree}{\text{\ding{174}}}
\newcommand{\circledFour}{\text{\ding{175}}}
\newcommand{\circledFive}{\text{\ding{176}}}
\newcommand{\circledSix}{\text{\ding{177}}}
\newcommand{\circledSeven}{\text{\ding{178}}}
\newcommand{\circledEight}{\text{\ding{179}}}
\newcommand{\circledNine}{\text{\ding{180}}}
\newcommand{\circledTen}{\text{\ding{181}}}

%PROBABILITY_TH
\newcommand{\cov}{\text{cov}}

%DISTRIBUTIONS
\newcommand{\Be}{\text{Be}}
\newcommand{\Binom}{\text{Binom}}
\newcommand{\Poisson}{\text{Poisson}}
\newcommand{\Geo}{\text{Geom}}
\newcommand{\NegBinom}{\text{NB}}
\newcommand{\Uniform}{{\mathcal{U}}}
\newcommand{\Normal}{{\mathcal{N}}}
\newcommand{\Exp}{\text{Exp}}
\newcommand{\Ca}{\text{Ca}}
\newcommand{\GammaDistr}{\text{Gamma}}
\newcommand{\ChiSquared}{\chi^2}
\newcommand{\BetaDistr}{\text{Beta}}
\newcommand{\LogNormal}{\log\Normal}

%GRAPHS
\newcommand{\rank}{\text{rank}}
\newcommand{\size}{\text{size}}

%ARRAYS
\newcommand{\length}{\text{length}}
\newcommand{\heapsize}{\text{heap-size}}

%DFT
\newcommand{\DFT}{\text{DFT}}
\newcommand{\FFT}{\text{FFT}}

%NUMBER THEORY
\newcommand{\GCD}[2]{\text{НОД}\left(#1, #2\right)}
\newcommand{\ord}[1]{\text{ord}\left(#1\right)}
\newcommand{\ind}[3]{\text{ind}_{#1, #2}\left(#3\right)}
\newcommand{\legandre}[2]{\left(\frac{#1}{#2}\right)}

%COMPLEXITY CLASSES
\def\DTIME{{\cal{D}\text{TIME}}}
\def\P{{\cal{P}}}
\def\EXPTIME{{\text{EXPTIME}}}
\def\NTIME{{\cal{N}\text{TIME}}}
\def\NP{{\cal{N}\cal{P}}}
\def\NEXPTIME{{\cal{N}\text{EXPTIME}}}
\def\coNP{{\text{co-}\cal{N}\cal{P}}}
\def\NPcomplete{{\cal{N}\cal{P}}\text{-complete}}
\def\NPhard{{\cal{N}\cal{P}}\text{-hard}}

%PD custom commands
\def\tnf{\widetilde{\nabla}f}
\def\tnmf{\widetilde{\nabla}^m f}
\newcommand{\la}{\langle}
\newcommand{\ra}{\rangle}
\def\e{\varepsilon}

\newcommand{\Var}{\text{Var}}

\def\eqdef{\overset{\text{def}}{=}}

\def \R {\mathbb R}

\begin{document}

\forloop{example}{0}{\value{example} < \value{totalpages}}


\begin{center}

\newcommand{\HRule}{\rule{\linewidth}{0.5mm}}
\HRule \\[0.2cm]
{ \Large \bfseries \topic} %\\[0.2cm]
\HRule
\end{center}

В этом домашнем задании можно набрать больше 10 баллов\footnote{Порог выставления зачёта ``автоматом'' при этом остаётся прежним.}.

\begin{center}
    \Large \bfseries Основные задачи
\end{center}

\begin{enumerate}
    \item (2 балла) Пусть $\eta$~--- случайный $n$-мерный вектор (например, стохастический градиент). Предположим, что $\EE\left[\|\eta\|_2^2\right] < \infty$ (второй момент $\eta$ ограничен). Пусть $\Var[\eta] = \EE\left[\left\|\eta - \EE\eta\right\|_2^2\right]$ (дисперсия случайного вектора $\eta$). Докажите, что $\EE\left[\|\eta\|_2^2\right] = \Var\left[\eta\right] + \left\|\EE\eta\right\|_2^2$.
    \item (2 балла) Рассмотрим задачу минимизации суммы функций:
    \begin{equation}
        f(x) = \frac{1}{m}\sum\limits_{i=1}^m f_i(x) \longrightarrow \min\limits_{x\in\R^n}. \label{eq:finite_sum_min}
    \end{equation}
    Пусть $\xi$~--- это случайная величина, которая случайно равновероятно принимает значения из множества $\{1,2,\ldots,m\}$. Как было показано на лекции, случайный вектор $\nabla f_{\xi}(x)$ является несмещённой оценкой градиента $f$, т.е.\ $\EE\left[\nabla f_{\xi}(x)\right] = \nabla f(x)$. Кроме того, были озвучены результаты о сходимости стохастического градиентного спуска ({\tt SGD}) в предположении ограниченности второго момента $\nabla f_{\xi}(x)$, т.е.\ в предположении, что существует такое число $M > 0$, что для всех $x\in\R^n$ выполняется $\EE\left[\|\nabla f_{\xi}(x)\|_2^2\right] \le M^2$. Покажите, что существует такая задача оптимизации \eqref{eq:finite_sum_min}, которая имеет решение, но для которой величина $\EE\left[\|\nabla f_{\xi}(x)\|_2^2\right]$ не является ограниченной, в то время как дисперсия $\nabla f_{\xi}(x)$ ограничена некоторой константой\footnote{Вообще говоря, во многих практически важных задачах второй момент стохастического градиента не ограничен константой и даже дисперсия не ограничена некоторой константой. Тем не менее долгое время для {\tt SGD} существовал анализ сходимости только в таких предположениях.}. 
    \item (1 балл) Как сводить одномерный поиск на полуинтервале, возникающий при использовании метода наискорейшего спуска для выпуклой функции, к одномерному поиску на отрезке? Предложите алгоритм, решающий эту задачу и оцените число итераций, необходимое процедуре одномерного поиска, если необходимая точность по аргументу равняется $\varepsilon$.
    \item (1 балл) Объясните, почему метод центров тяжести трудно эффективно реализовать на практике.
    \item (2+2 балла) Рассмотрим задачу
    \begin{equation}
        f(x) \longrightarrow \min\limits_{x\in Q\subseteq \R^n}, \label{eq:problem_gen}
    \end{equation}
    где $Q$~--- выпуклое замкнутое подмножество $\R^n$, функция $f(x)$ выпукла и дифференцируема на $Q$, причём градиент $f$~--- Липшицева на множестве $Q$ функция с константой Липшица $L>0$, т.е.\ для всех $x,y\in Q$ выполняется неравенство
    $$
        \|\nabla f(x) - \nabla f(y)\|_2 \le L\|x - y\|_2.
    $$
    \begin{enumerate}
        \item Докажите\footnote{\textit{Подсказка:} воспользуйтесь представлением $f(y) - f(x) = \int\limits_{0}^1 \left\langle\nabla f(x + t(y-x)), y - x\right\rangle dt$.}, что для всех $x,y\in Q$
        \begin{equation}
            f(y) \le f(x) + \langle\nabla f(x),y-x \rangle + \frac{L}{2}\|x-y\|_2^2\notag.
        \end{equation}
        \item Для решения \eqref{eq:problem_gen} на семинаре был рассмотрен метод проекции градиента
        \begin{equation*}
            x^{k+1} = \pi_{Q}\left(x^k - \frac{1}{L}\nabla f(x^k)\right),
        \end{equation*}
        где $\pi_Q(\cdot)$ определяется как 
        \begin{equation*}
            \pi_Q(y) = \argmin\limits_{x\in Q}\|x - y\|_2.
        \end{equation*}
        Докажите, что
        \begin{equation*}
            \argmin\limits_{x\in Q}\left\{f(x^k) + \langle\nabla f(x^k),x-x^k\rangle+\frac{L}{2}\|x-x^k\|_2^2\right\} = \pi_{Q}\left(x^k - \frac{1}{L}\nabla f(x^k)\right).
        \end{equation*}
    \end{enumerate}
\end{enumerate}

\begin{center}
    \Large \bfseries Дополнительные задачи
\end{center}

\begin{enumerate}
    \item (2+2 балла) Рассмотрим задачу
    \begin{equation}
        f(x) \longrightarrow \min\limits_{x\in[a,b]\in \R}, \label{eq:line_search_problem}
    \end{equation}
    где $f$~--- выпуклая дифференцируемая функция, причём $|f'(x)| \le M$ для всех $x\in[a,b]$.
    \begin{enumerate}
        \item Предположим, что для решения этой задачи мы применяем метод деления отрезка пополам, использующий вместо производной $\delta$-производную $\tilde{f}'(x)$ ($\delta$~--- некоторое положительное число), которая в каждой точке $x\in[a,b]$ удовлетворяет неравенству
    \begin{equation*}
        |\tilde f'(x) - f'(x)| \le \delta.
    \end{equation*}
    Предположим, что через $N$ итераций метода деления отрезка пополам мы получили некоторую точку $x^N$. Оцените $f(x^N) - f(x^*)$ через $N, b-a, \delta$ и $M$, где $x^* = \argmin_{x\in [a,b]} f(x)$.
        \item Предположим теперь, что нам доступны только $\delta$-зашумлённые значения функции $\tilde{f}(x)$, т.е.\ для точки $x\in[a,b]$ оракул выдаёт нам такое число $\tilde{f}(x)$, что $$|\tilde f(x) - f(x)| \le \delta.$$
        Пусть $x^N$~--- это точка, которую выдаёт метод золотого сечения, использующий значения $\tilde f(x)$ вместо $f(x)$. Оцените $f(x^N) - f(x^*)$ через $N, b-a, \delta$ и $M$, где $x^* = \argmin_{x\in [a,b]} f(x)$.
    \end{enumerate}
    \item (2 балла) Используя рассуждения на стр. 232-233 \href{https://arxiv.org/ftp/arxiv/papers/1711/1711.00394.pdf}{пособия} при $n = 1$  предложите способ ускорения локальной скорости сходимости линейно сходящихся методов одномерного поиска в сильно выпуклом случае.
    \item (4 балла) Рассмотрим задачу
    \begin{equation*}
        f(x) \longrightarrow \min\limits_{x\in Q\in\R^n},
    \end{equation*}
    где $f(x)$~--- выпуклая дифференцируемая на $Q$ функция, $Q$~--- выпуклое замкнутое подмножество $\R^n$ с непустой внутренностью, причём существуют такие положительные числа $r$ и $R$, что $Q$ содержит некоторый шар радиуса $r$ и $Q$ содержится в некотором шаре радиуса $R$. Предположим, что существует такая константа $B>0$, что
    $$
        \forall x\in Q\quad |f(x)| \le B.
    $$
    Для небольших размерностей $n$ на лекции был описан метод, позволяющий быстро решать такую задачу~--- метод эллипсоидов. В основе метода эллипсоидов лежит оракул разделения, который в нашем случае требует вычисления градиента функции $f$ (или её субградиента, но для простоты будем считать, что функция $f$ дифференцируема). Предположим, что нет возможности точно вычислить градиент $f$, но вместо этого в любой точке метод имеет доступ к $\delta$-градиенты $f$, т.е.\ для точки $x\in Q$ оракул выдаёт такой вектор $\tilde\nabla f(x)$, что
    $$
        \|\tilde\nabla f(x) - \nabla f(x)\| \le \delta.
    $$
    Пусть через $N$ итерация метод эллипсоидов, использующий $\delta$-градиент вместо точного градиента, вернул точку $x^N$. Оцените $f(x^N) - f(x^*)$ через $N, n, R, r, B, \delta$, где $x^* = \argmin_{x\in Q} f(x)$. За основу можно взять рассуждения из \href{http://sbubeck.com/Bubeck15.pdf}{обзора Себастьяна Бубека} (стр. 20-23).
\end{enumerate}

\end{document}