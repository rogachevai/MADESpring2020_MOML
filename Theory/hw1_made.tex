\documentclass[a4paper,12pt]{article}
\pagestyle{empty}
\usepackage[T2A]{fontenc}
\usepackage[utf8]{inputenc}
\usepackage[russian]{babel}
\usepackage{cmap}
\usepackage{amsthm}
\usepackage{amsmath}
\usepackage{units}
\usepackage{fancyhdr}
\usepackage{forloop}
\usepackage{amssymb}
\usepackage{url}
\usepackage{hyperref}
\usepackage{xcolor}
\usepackage{color}
\usepackage{pifont}
\usepackage{graphicx}
\usepackage{subfig}
\definecolor{urlcolor}{HTML}{0645AD}
\hypersetup{pdfstartview=FitH, linkcolor=black,urlcolor=urlcolor, colorlinks=true}
\usepackage{algorithm}
\usepackage{algpseudocode}

\usepackage{tikz}

\usepackage{verbatim}
\usetikzlibrary{arrows,shapes}

\theoremstyle{definition}

\newtheorem{theorem}{Теорема}
\newtheorem{lemma}{Лемма}
\newtheorem{definition}{Определение}
\newtheorem{corollary}{Следствие}
\newtheorem{example}{Пример}
\newtheorem{remark}{Замечание}
\newtheorem{exercise}{Упражнение}

\renewcommand{\thesection}{\arabic{section}}

\renewcommand{\headrulewidth}{0.4pt}
\renewcommand{\footrulewidth}{0.4pt}

\fancyhead[RO]{\bfseries \rightmark}
\fancyhead[LE]{\bfseries \rightmark}
\fancyhead[LO, RE]{Методы оптимизации в машинном обучении}
\fancyfoot[L]{MADE 2020}
%For multipage documents only!
\fancyhead[R]{Страница \thepage}
%Uncomment this for 1-page sheets
\fancyfoot[R]{}
\fancyfoot[C]{}

\pagestyle{fancy}

\renewcommand{\baselinestretch}{1.0}
\renewcommand\normalsize{\sloppypar}

\setlength{\topmargin}{-0.1in}
\setlength{\textheight}{9.1in}
\setlength{\oddsidemargin}{-0.3in}
\setlength{\evensidemargin}{-0.3in}
\setlength{\textwidth}{7in}
\setlength{\parindent}{0ex}
\setlength{\parskip}{1ex}

\newenvironment{comm}{}{}

\newcounter{problemset}
\newcounter{totalpages}
%Here you should set the total number of pages
\setcounter{totalpages}{1}

\def \topic {Домашнее задание №1. \\Введение/Выпуклые множества/Выпуклые функции. \\
Векторное дифференцирование.}

\def \ZZ {\mathbb Z}
\def \FF {\mathbb F}
%\def \RR {\mathbb R}
\def \QQ {\mathbb Q}
\def \NN {\mathbb N}
\def \PP {\mathbb P}
\def \EE {\mathbb E}
\def \DD {\mathbb D}
\def \CC {\mathbb C}

\def\cA{{\cal A}}
\def\cB{{\cal B}}
\def\cD{{\cal D}}
\def\cC{{\cal C}}
\def\cQ{{\cal Q}}
%\def\R{{\cal R}}
\def\cM{{\cal M}}
\def\cN{{\cal N}}
\def\cT{{\cal T}}
\def\cP{{\cal P}}
\def\cF{{\cal F}}

\def\cE{{\mathcal{E}}}

\def\poly{{\text{poly}}}

\newcommand{\argmin}{\mathop{\arg\!\min}}
\newcommand{\circledOne}{\text{\ding{172}}}
\newcommand{\circledTwo}{\text{\ding{173}}}
\newcommand{\circledThree}{\text{\ding{174}}}
\newcommand{\circledFour}{\text{\ding{175}}}
\newcommand{\circledFive}{\text{\ding{176}}}
\newcommand{\circledSix}{\text{\ding{177}}}
\newcommand{\circledSeven}{\text{\ding{178}}}
\newcommand{\circledEight}{\text{\ding{179}}}
\newcommand{\circledNine}{\text{\ding{180}}}
\newcommand{\circledTen}{\text{\ding{181}}}

%PROBABILITY_TH
\newcommand{\cov}{\text{cov}}

%DISTRIBUTIONS
\newcommand{\Be}{\text{Be}}
\newcommand{\Binom}{\text{Binom}}
\newcommand{\Poisson}{\text{Poisson}}
\newcommand{\Geo}{\text{Geom}}
\newcommand{\NegBinom}{\text{NB}}
\newcommand{\Uniform}{{\mathcal{U}}}
\newcommand{\Normal}{{\mathcal{N}}}
\newcommand{\Exp}{\text{Exp}}
\newcommand{\Ca}{\text{Ca}}
\newcommand{\GammaDistr}{\text{Gamma}}
\newcommand{\ChiSquared}{\chi^2}
\newcommand{\BetaDistr}{\text{Beta}}
\newcommand{\LogNormal}{\log\Normal}

%GRAPHS
\newcommand{\rank}{\text{rank}}
\newcommand{\size}{\text{size}}

%ARRAYS
\newcommand{\length}{\text{length}}
\newcommand{\heapsize}{\text{heap-size}}

%DFT
\newcommand{\DFT}{\text{DFT}}
\newcommand{\FFT}{\text{FFT}}

%NUMBER THEORY
\newcommand{\GCD}[2]{\text{НОД}\left(#1, #2\right)}
\newcommand{\ord}[1]{\text{ord}\left(#1\right)}
\newcommand{\ind}[3]{\text{ind}_{#1, #2}\left(#3\right)}
\newcommand{\legandre}[2]{\left(\frac{#1}{#2}\right)}

%COMPLEXITY CLASSES
\def\DTIME{{\cal{D}\text{TIME}}}
\def\P{{\cal{P}}}
\def\EXPTIME{{\text{EXPTIME}}}
\def\NTIME{{\cal{N}\text{TIME}}}
\def\NP{{\cal{N}\cal{P}}}
\def\NEXPTIME{{\cal{N}\text{EXPTIME}}}
\def\coNP{{\text{co-}\cal{N}\cal{P}}}
\def\NPcomplete{{\cal{N}\cal{P}}\text{-complete}}
\def\NPhard{{\cal{N}\cal{P}}\text{-hard}}

%PD custom commands
\def\tnf{\widetilde{\nabla}f}
\def\tnmf{\widetilde{\nabla}^m f}
\newcommand{\la}{\langle}
\newcommand{\ra}{\rangle}
\def\e{\varepsilon}

\def\eqdef{\overset{\text{def}}{=}}

\def \R {\mathbb R}

\begin{document}

\forloop{example}{0}{\value{example} < \value{totalpages}}


\begin{center}

\newcommand{\HRule}{\rule{\linewidth}{0.5mm}}
\HRule \\[0.2cm]
{ \Large \bfseries \topic} %\\[0.2cm]
\HRule

\end{center}

\begin{flushleft}
	\textbf{ВВЕДЕНИЕ (3 балла)} 
\end{flushleft}

\begin{enumerate}
    \item Для успешного выступления на соревнованиях спортсмен должен потратить как минимум $\mathrm{K}$ килокалорий. Вечером перед соревнованиями он идёт в магазин, чтобы купить продукты себе на ужин. В магазине представлено m наименований товаров, цена единицы каждого наименования равна $p_i$, $i = 1, . . . , m$. Также известно, что $i$-ая единица каждого товара придаёт студенту энергию равную $k_i$ килокалорий. Поставьте задачу определения содержимого корзины минимальной стоимости для успешного выступления. Является ли поставленная задача выпуклой и почему?
    
    \item Что такое норма вектора? Что такое эквивалентность норм? Показать эквивалентность $l_1,l_2, l_{\infty}$ норм.
    
    \item Что такое норма матрицы (линейного конечномерного оператора)? Чему равны $\mathrm{L}_1, \mathrm{L}_2,
    \mathrm{L}_{\infty}$ нормы матрицы? Что такое Фробениусова норма матрицы и какой аналог среди векторных норм векторов она имеет?
\end{enumerate}

\begin{flushleft}
	\textbf{ВЫПУКЛЫЕ/КОНИЧЕСКИЕ МНОЖЕСТВА (2 балла)}

	\begin{enumerate}
          \item Доказать по определению, что гиперболическое множество
            \[ \left\{x \in \mathbb{R}^n_+ \ \bigg| \ \prod^{n}_{i=1}x_i \geq 1\right\} \text{-- выпуклое.}\]
  
  \item Пусть $\mathrm{P} \in \mathbb{S}^n_{++}$, $c \in \R^n$. Покажите, что множество \[\left\{ x \in \R^n \ | \ \langle \mathrm{P} x, x \rangle \leq \langle c, x \rangle^2, \ \langle c, x \rangle \geq 0 \right\}\text{-- выпуклое.}\] 
  
  
\end{enumerate}
\end{flushleft}

\begin{flushleft}
	\textbf{ВЫПУКЛЫЕ ФУНКЦИИ (2 балла)}
\end{flushleft}	
\begin{enumerate}
    \item 
	Докажите выпуклость следующих функций:
\begin{enumerate}
    \item $f(x) = \left(\sum\limits_{i=1}^n e^{x_i} \right)$;
    \item $f(x) = \frac{\|\mathrm{A}x-b\|^2}{1 - x^{\top}x}, \ \mathrm{X} = \{x \ | \ \|x\|^2 \leq 1\}$;
    \item $F(x) = \frac{f^2(x)}{g(x)}$ при условии, что $f(x)$ является выпуклой и принимает только неотрицательные значения, $g(x)$ вогнута и принимает только положительные значения;
    \item $f(x,t) = - \log (t^2 - x^Tx), \ \mathrm{E} = \{(x,t)\} \in \R^{n \times 1} \ | \ \|x\|_2<t \}$;
    \item $f(x) = \sum\limits_{i=1}^n w_i \ln \left( 1 + \exp (a_i^{\top} x)\right) + \frac{\mu}{2} \|x\|_2^2, \; \ \mu, w_1, \dots, w_n > 0, \;  a_i \in \mathbb{R}^n$;
    \item $f(x) = \ln \sum\limits_{i=1}^n \exp \left( \max \{0, x_i \}^2\right)$.
\end{enumerate}

\end{enumerate}

\begin{flushleft}
	\textbf{ТЕСТ/ВЫПУКЛЫЕ ФУНКЦИИ (2 балла)}
\end{flushleft}	


\begin{enumerate}
\item Какое максимальное количество точек минимума может быть у выпуклой функции? 
\begin{enumerate}
    \item Одна
    \item Две
    \item Счётное множество
    \item Несчётное множество
\end{enumerate} 

\item Какие из следующих функций выпуклы? 
\begin{enumerate}
    \item $f(x) = \sum_{i=1}^r x_{[i]}$, где $x_{[1]} \geq x_{[1]} \geq \ldots \geq x_{[n]}$~--- упорядоченные по убыванию элементы вектора $x$
    \item $f(x) = x^5$
    \item $f(x) = \sum_{i=1}^n x_i \log x_i$, где $x_i \geq 0$ и $f(0) = 0$ по непрерывности
    \item $f(A) = x^{\top}Ax$, где $A \in \mathbb{R}^{n \times n}$, $x \in \mathbb{R}^n$~--- фиксированный вектор
    \item $\|x\|_{1/3} = (\sum_{i=1}^n |x_i|^{1/3})^3$
\end{enumerate}

\item Чем является касательная к графику выпуклой функции в каждой точке области определения?
\begin{enumerate}
\item Локальной оценкой сверху
\item Глобальной оценкой снизу
\item Локальной оценкой снизу
\item Глобальной оценкой сверху
\end{enumerate}

\item На каком множестве выпукла функция $f =  x^3_2 + x_2(x^2_1 + x^2_3)$?
\begin{enumerate}
\item $\{ (x_1, x_2, x_3) \; | \; x_1 \geq 0, \; x_2 \geq 0, \; x_3 \geq 0 \}$
\item $\{ (x_1, x_2, x_3)\; | \; x_2 \leq 0, \; 3x_2^2 - x_1^2 \geq 0 \}$
\item $\{ (x_1, x_2, x_3)\; | \; x_2 \geq 0, \; 3x_2^2 - x_1^2 \geq x_3^2 \}$
\item $\{ (x_1, x_2, x_3) \; | \; x_2 \geq 0, \; 6x_2^2 - x_1^2 \geq 0 \}$
\end{enumerate}

\item Может ли область определения выпуклой функции быть невыпуклым множеством?
 \begin{enumerate}
\item Да
\item Нет
\end{enumerate}

\item Отметьте все композиции $f(x) = h(g(x))$, такие что $f(x)$~--- выпуклая функция.
\begin{enumerate}
\item $h(x) = \log x$, $g(x)$~-- выпуклая функция
\item $h(x) = e^x$, $g(x)$~-- вогнутая функция
\item $g(x) = Ax + b$ для некоторой матрицы $A$ и вектора $b$, $h$~-- выпуклая функция
\item $h(x) = |x - 4|$, $g(x) = |x|$
\end{enumerate}

\item Чему равна константа сильной выпуклости у функции $f(x) = \frac12\|x\|_2^2$? 
\begin{enumerate}
\item 1
\item 2
\item 3
\item 10
\end{enumerate}
\end{enumerate}

\begin{flushleft}
	\textbf{ТЕСТ/ВЕКТОРНОЕ ДИФФЕРЕНЦИРОВАНИЕ (2 балла)}
\end{flushleft}	


\begin{enumerate}
\item 
Посчитайте $df(x)$ и $\nabla f(x)$ для функции $f(x) = \log(x^{\top}\mathrm{A}x)$ и выберите правильный вариант ответа.
    \begin{enumerate}
    \item $\nabla f(x) = \frac{2\left(\mathrm{A}+ \mathrm{A}^{\top} \right)x}{x^{\top}\mathrm{A}x}, \; df(x) = \frac{x^{\top}2\left(\mathrm{A}+ \mathrm{A}^{\top} \right)dx}{x^{\top}\mathrm{A}x}$
    \item $\nabla f(x) = \frac{\left(\mathrm{A}+ \mathrm{A}^{\top} \right)x}{x^{\top}\mathrm{A}x}, \; df(x) = \frac{x^{\top}\left(\mathrm{A}+ \mathrm{A}^{\top} \right)dx}{x^{\top}\mathrm{A}x}$
    \item $\nabla f(x) = \frac{x^{\top}\left(\mathrm{A}+ \mathrm{A}^{\top} \right)dx}{x^{\top}\mathrm{A}x}, \; d f(x) = \frac{\left(\mathrm{A}+ \mathrm{A}^{\top} \right)x}{x^{\top}\mathrm{A}x}$
    \item $\nabla f(x) = \frac{x^{\top}2\left(\mathrm{A}+ \mathrm{A}^{\top} \right)dx}{x^{\top}\mathrm{A}x}, \; d f(x) = \frac{2\left(\mathrm{A}+ \mathrm{A}^{\top} \right)x}{x^{\top}\mathrm{A}x}$
    \end{enumerate}
    
\item Посчитайте $df(x), \; d^2f(x)$ и $\nabla f(x), \; \nabla^2 f(x)$ для функции $f(x) = \frac1p \|x\|^p_2, \; p>1$ и выберите все правильные результаты.
    \begin{enumerate}
    \item $\nabla f(x) = \|x\|_2^{p-1}x$
    \item $df(x) = \|x\|_2^{p-2} x^{\top}dx$
    \item $\nabla^2 f(x) =  (p-2) \|x\|_2^{p-4}x^{\top}x + \|x\|_2^{p-2} $
    \item $\nabla^2 f(x) =  (p-1) \|x\|_2^{p-3}xx^{\top} + \|x\|_2^{p-1} I$
    \item $d^2f(x) = dx^{\top} \left( (p-2) \|x\|_2^{p-4}xx^{\top} + \|x\|_2^{p-2} I\right) dx$
    \item $d^2f(x) = dx^{\top} \left( (p-1) \|x\|_2^{p-3}xx^{\top} + \|x\|_2^{p-1} I\right) dx$
    \end{enumerate}
    
\item Верно ли, что у функции 
$f(x) = \frac1n \sum\limits_{i=1}^n \log \left( 1 + \exp(a_i^{\top}x)  \right) + \frac{\mu}{2}\|x\|_2^2, \; a_i \in \mathbb R^n, \; \mu>0, \;$ гессиан $\nabla^2 f(x) = \frac1n \sum\limits_{i=1}^n \left( -\frac{\exp(a_i^{\top}x)}{(1 + \exp(a_i^{\top}x))^2}a_ia_i^{\top} + \frac{\exp(a_i^{\top}x)}{1 + \exp(a_i^{\top}x)}a_ia_i^{\top}\right) + \mu \mathrm{I}$?
\begin{enumerate}
\item Да
\item Нет
\end{enumerate}
\end{enumerate}

\end{document}